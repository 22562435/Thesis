\documentclass[11pt,preprint]{elsarticle}

\usepackage{lmodern}
%%%% My spacing
\usepackage{setspace}
\setstretch{1.2}
\DeclareMathSizes{12}{14}{10}{10}

% Wrap around which gives all figures included the [H] command, or places it "here". This can be tedious to code in Rmarkdown.
\usepackage{float}
\let\origfigure\figure
\let\endorigfigure\endfigure
\renewenvironment{figure}[1][2] {
    \expandafter\origfigure\expandafter[H]
} {
    \endorigfigure
}

\let\origtable\table
\let\endorigtable\endtable
\renewenvironment{table}[1][2] {
    \expandafter\origtable\expandafter[H]
} {
    \endorigtable
}


\usepackage{ifxetex,ifluatex}
\usepackage{fixltx2e} % provides \textsubscript
\ifnum 0\ifxetex 1\fi\ifluatex 1\fi=0 % if pdftex
  \usepackage[T1]{fontenc}
  \usepackage[utf8]{inputenc}
\else % if luatex or xelatex
  \ifxetex
    \usepackage{mathspec}
    \usepackage{xltxtra,xunicode}
  \else
    \usepackage{fontspec}
  \fi
  \defaultfontfeatures{Mapping=tex-text,Scale=MatchLowercase}
  \newcommand{\euro}{€}
\fi

\usepackage{amssymb, amsmath, amsthm, amsfonts}

\def\bibsection{\section*{References}} %%% Make "References" appear before bibliography


\usepackage[numbers]{natbib}

\usepackage{longtable}
\usepackage[margin=2.3cm,bottom=2cm,top=2.5cm, includefoot]{geometry}
\usepackage{fancyhdr}
\usepackage[bottom, hang, flushmargin]{footmisc}
\usepackage{graphicx}
\numberwithin{equation}{section}
\numberwithin{figure}{section}
\numberwithin{table}{section}
\setlength{\parindent}{0cm}
\setlength{\parskip}{1.3ex plus 0.5ex minus 0.3ex}
\usepackage{textcomp}
\renewcommand{\headrulewidth}{0.2pt}
\renewcommand{\footrulewidth}{0.3pt}

\usepackage{array}
\newcolumntype{x}[1]{>{\centering\arraybackslash\hspace{0pt}}p{#1}}

%%%%  Remove the "preprint submitted to" part. Don't worry about this either, it just looks better without it:
\makeatletter
\def\ps@pprintTitle{%
  \let\@oddhead\@empty
  \let\@evenhead\@empty
  \let\@oddfoot\@empty
  \let\@evenfoot\@oddfoot
}
\makeatother

 \def\tightlist{} % This allows for subbullets!

\usepackage{hyperref}
\hypersetup{breaklinks=true,
            bookmarks=true,
            colorlinks=true,
            citecolor=blue,
            urlcolor=blue,
            linkcolor=blue,
            pdfborder={0 0 0}}


% The following packages allow huxtable to work:
\usepackage{siunitx}
\usepackage{multirow}
\usepackage{hhline}
\usepackage{calc}
\usepackage{tabularx}
\usepackage{booktabs}
\usepackage{caption}


\newenvironment{columns}[1][]{}{}

\newenvironment{column}[1]{\begin{minipage}{#1}\ignorespaces}{%
\end{minipage}
\ifhmode\unskip\fi
\aftergroup\useignorespacesandallpars}

\def\useignorespacesandallpars#1\ignorespaces\fi{%
#1\fi\ignorespacesandallpars}

\makeatletter
\def\ignorespacesandallpars{%
  \@ifnextchar\par
    {\expandafter\ignorespacesandallpars\@gobble}%
    {}%
}
\makeatother


% definitions for citeproc citations
\NewDocumentCommand\citeproctext{}{}
\NewDocumentCommand\citeproc{mm}{%
\href{\#cite.\detokenize{#1}}{#2}\nocite{#1}}

\makeatletter
% allow citations to break across lines
\let\@cite@ofmt\@firstofone
% avoid brackets around text for \cite:
\def\@biblabel#1{}
\def\@cite#1#2{{#1\if@tempswa , #2\fi}}
\makeatother
\newlength{\cslhangindent}
\setlength{\cslhangindent}{1.5em}
\newlength{\csllabelwidth}
\setlength{\csllabelwidth}{3em}
\newenvironment{CSLReferences}[2] % #1 hanging-indent, #2 entry-spacing
{\begin{list}{}{%
	\setlength{\itemindent}{0pt}
	\setlength{\leftmargin}{0pt}
	\setlength{\parsep}{0pt}
	% turn on hanging indent if param 1 is 1
	\ifodd #1
	\setlength{\leftmargin}{\cslhangindent}
	\setlength{\itemindent}{-1\cslhangindent}
	\fi
	% set entry spacing
	\setlength{\itemsep}{#2\baselineskip}}}
{\end{list}}

\usepackage{calc}
\newcommand{\CSLBlock}[1]{\hfill\break\parbox[t]{\linewidth}{\strut\ignorespaces#1\strut}}
\newcommand{\CSLLeftMargin}[1]{\parbox[t]{\csllabelwidth}{\strut#1\strut}}
\newcommand{\CSLRightInline}[1]{\parbox[t]{\linewidth - \csllabelwidth}{\strut#1\strut}}
\newcommand{\CSLIndent}[1]{\hspace{\cslhangindent}#1}


\urlstyle{same}  % don't use monospace font for urls
\setlength{\parindent}{0pt}
\setlength{\parskip}{6pt plus 2pt minus 1pt}
\setlength{\emergencystretch}{3em}  % prevent overfull lines
\setcounter{secnumdepth}{5}

%%% Use protect on footnotes to avoid problems with footnotes in titles
\let\rmarkdownfootnote\footnote%
\def\footnote{\protect\rmarkdownfootnote}
\IfFileExists{upquote.sty}{\usepackage{upquote}}{}

%%% Include extra packages specified by user

%%% Hard setting column skips for reports - this ensures greater consistency and control over the length settings in the document.
%% page layout
%% paragraphs
\setlength{\baselineskip}{12pt plus 0pt minus 0pt}
\setlength{\parskip}{12pt plus 0pt minus 0pt}
\setlength{\parindent}{0pt plus 0pt minus 0pt}
%% floats
\setlength{\floatsep}{12pt plus 0 pt minus 0pt}
\setlength{\textfloatsep}{20pt plus 0pt minus 0pt}
\setlength{\intextsep}{14pt plus 0pt minus 0pt}
\setlength{\dbltextfloatsep}{20pt plus 0pt minus 0pt}
\setlength{\dblfloatsep}{14pt plus 0pt minus 0pt}
%% maths
\setlength{\abovedisplayskip}{12pt plus 0pt minus 0pt}
\setlength{\belowdisplayskip}{12pt plus 0pt minus 0pt}
%% lists
\setlength{\topsep}{10pt plus 0pt minus 0pt}
\setlength{\partopsep}{3pt plus 0pt minus 0pt}
\setlength{\itemsep}{5pt plus 0pt minus 0pt}
\setlength{\labelsep}{8mm plus 0mm minus 0mm}
\setlength{\parsep}{\the\parskip}
\setlength{\listparindent}{\the\parindent}
%% verbatim
\setlength{\fboxsep}{5pt plus 0pt minus 0pt}



\begin{document}



\begin{frontmatter}  %

\title{Thesis Proposal: Forecasting JSE Equity Volatility (ML
vs.~GARCH/HAR Models)}

% Set to FALSE if wanting to remove title (for submission)




\author[Add1]{Liam Andrew Beattie}
\ead{22562435@sun.ac.za}





\address[Add1]{Stellenbosch University, South Africa}


\begin{abstract}
\small{
Compare the performance of Machine Learning (ML) models (e.g., LSTM,
XGBoost) and traditional econometric models (GARCH, HAR) in forecasting
equity volatility on the Johannesburg Stock Exchange (JSE),
contextualized to South Africa's unique market dynamics (e.g., commodity
dependence, political risk).
}
\end{abstract}

\vspace{1cm}





\vspace{0.5cm}

\end{frontmatter}

\setcounter{footnote}{0}



%________________________
% Header and Footers
%%%%%%%%%%%%%%%%%%%%%%%%%%%%%%%%%
\pagestyle{fancy}
\chead{}
\rhead{}
\lfoot{}
\rfoot{\footnotesize Page \thepage}
\lhead{}
%\rfoot{\footnotesize Page \thepage } % "e.g. Page 2"
\cfoot{}

%\setlength\headheight{30pt}
%%%%%%%%%%%%%%%%%%%%%%%%%%%%%%%%%
%________________________

\headsep 35pt % So that header does not go over title




\section{\texorpdfstring{Introduction
\label{Introduction}}{Introduction }}\label{introduction}

Here are the papers I intend to read and give a brief summary of them:

Gunnarsson, Isern, Kaloudis, Risstad, Vigdel \& Westgaard
(\citeproc{ref-GUNNARSSON2024103221}{2024})

Notes that ML and AI methods rank high and on par with econometric
predicitons of realised volatility. Using explainable AI or creating
explainable AI might be interesting in the context of my paper.

This paper looks at realised volatility which is inferreed from the sum
of square intradaily hihg-frequency returns.

\begin{quote}
We find that, going forward, the use of XAI and implied volatility
forecasting forecasting proves an interesting direction for further
research, as well as combination of the promsing performing models and
Bayesian approaches to quanitify the uncertainty introduced by models to
explain trustworthiness of predictions.
\end{quote}

The top three testing messures found in Gunnarsson \emph{et al.}
(\citeproc{ref-GUNNARSSON2024103221}{2024}) to compare models and
forecasting was - Diebold-Mariano - Clark-West - Model Confidence Set

On a slight tangent, this paper references another paper by Bouri et
al.~(2020) who used google search volume intensity associated with the
US-China trade war to capture uncertainty aspects. In terms of data
input, if we could get access to this, would be extremely valuable to
add some google data to my research.

Horvath, Muguruza \& Tomas (\citeproc{ref-Horvath2021DeepVol}{2021})
Petrozziello, Troiano, Serra, Jordanov, Storti, Tagliaferri \& Rocca
(\citeproc{ref-Petrozziello2022DeepVol}{2022}) Wu, Cheng, Jankovic \&
Kolanovic (\citeproc{ref-Wu2024InvestableAI}{2024}) Wu, Jankovic, Kaplan
\& Lee (\citeproc{ref-Wu2025CrossAssetVol}{2025}) Zeng \& Klabjan
(\citeproc{ref-ZENG2019376}{2019}) Zhang, Zhang, Cucuringu \& Qian
(\citeproc{ref-Zhang2023VolForecastML}{2023})

Zhang \emph{et al.} (\citeproc{ref-Zhang2023VolForecastML}{2023})

Compare the performance of \textbf{Machine Learning (ML)} models (e.g.,
LSTM, XGBoost) and \textbf{traditional econometric models} (GARCH, HAR)
in forecasting equity volatility on the Johannesburg Stock Exchange
(JSE), contextualized to South Africa's unique market dynamics (e.g.,
commodity dependence, political risk).

\section{Key Steps}\label{key-steps}

\subsection{Data Collection \&
Preparation}\label{data-collection-preparation}

\begin{itemize}
\tightlist
\item
  \textbf{Data Sources}:

  \begin{itemize}
  \tightlist
  \item
    \textbf{JSE Equity Data}: Daily/index returns (e.g., JSE Top 40)
    from Bloomberg, Yahoo Finance, or \texttt{Quantmod} in R.\\
  \item
    \textbf{Volatility Proxy}: Realized volatility (using intraday data)
    or daily squared returns.\\
  \item
    \textbf{Exogenous Variables}:

    \begin{itemize}
    \tightlist
    \item
      Commodity prices (platinum, gold) via \texttt{Quandl}.\\
    \item
      SA-specific risks (Eskom load-shedding schedules, political risk
      indices).\\
    \item
      Global factors (USD/ZAR exchange rate, Fed rates).
    \end{itemize}
  \end{itemize}
\item
  \textbf{Preprocessing}:

  \begin{itemize}
  \tightlist
  \item
    Handle missing data, outliers, and structural breaks (e.g.,
    COVID-19, 2021 riots).\\
  \item
    Normalize/standardize features for ML models.
  \end{itemize}
\end{itemize}

\subsection{Model Implementation}\label{model-implementation}

\begin{itemize}
\tightlist
\item
  \textbf{Traditional Models}:

  \begin{itemize}
  \tightlist
  \item
    \textbf{GARCH}: Use \texttt{rugarch} in R to model volatility
    clustering (e.g., EGARCH for asymmetry).\\
  \item
    \textbf{HAR-RV}: Implement Heterogeneous Autoregressive model with
    \texttt{highfrequency} package.\\
  \end{itemize}
\item
  \textbf{ML Models}:

  \begin{itemize}
  \tightlist
  \item
    \textbf{LSTM/GRU}: Sequence-based models using
    \texttt{keras}/\texttt{tensorflow}.\\
  \item
    \textbf{Tree-Based}: XGBoost (\texttt{xgboost}), Random Forest
    (\texttt{randomForest}).\\
  \item
    \textbf{Hybrids}: Combine HAR residuals with ML predictions.
  \end{itemize}
\end{itemize}

\subsection{Feature Engineering}\label{feature-engineering}

\begin{itemize}
\tightlist
\item
  \textbf{Lag Features}: Historical volatility, returns, trading
  volume.\\
\item
  \textbf{SA-Specific Features}:

  \begin{itemize}
  \tightlist
  \item
    Load-shedding stages (scraped via \texttt{rvest}).\\
  \item
    SA sovereign credit rating changes.\\
  \item
    Commodity price shocks (platinum/gold).
  \end{itemize}
\end{itemize}

\subsection{Training \& Validation}\label{training-validation}

\begin{itemize}
\tightlist
\item
  \textbf{Train-Test Split}: Time-series cross-validation (e.g., rolling
  window).\\
\item
  \textbf{Hyperparameter Tuning}: Grid search for ML models (e.g.,
  \texttt{mlr3} in R).
\end{itemize}

\subsection{Evaluation}\label{evaluation}

\begin{itemize}
\tightlist
\item
  \textbf{Metrics}:

  \begin{itemize}
  \tightlist
  \item
    Statistical: MSE, MAE, QLIKE.\\
  \item
    Economic: Value-at-Risk (VaR) backtesting, Sharpe ratio of
    volatility-based strategies.\\
  \end{itemize}
\item
  \textbf{Statistical Tests}:

  \begin{itemize}
  \tightlist
  \item
    Diebold-Mariano test for forecast superiority.\\
  \item
    Model Confidence Set (MCS) to rank models.
  \end{itemize}
\end{itemize}

\subsection{Interpretation \& Context}\label{interpretation-context}

\begin{itemize}
\tightlist
\item
  \textbf{XAI}: Use SHAP values (\texttt{DALEX}) to interpret ML drivers
  (e.g., ``How much do platinum prices impact mining stock
  forecasts?'').\\
\item
  \textbf{Comparative Analysis}: Contrast results with global markets
  (e.g., ``Why does XGBoost underperform in SA vs.~the S\&P 500?'').
\end{itemize}

\section{Deliverables}\label{deliverables}

\begin{enumerate}
\def\labelenumi{\arabic{enumi}.}
\tightlist
\item
  \textbf{Visualizations}:

  \begin{itemize}
  \tightlist
  \item
    Time-series plots of volatility with crisis annotations.\\
  \item
    SHAP summary plots for feature importance.\\
  \item
    Heatmaps of cross-correlations between variables.\\
  \end{itemize}
\item
  \textbf{Code}: Reproducible R scripts for data, models, and visuals.\\
\item
  \textbf{Policy Insights}: Practical recommendations for SA
  investors/policymakers (e.g., hedging strategies during
  load-shedding).
\end{enumerate}

\section{Challenges \& Solutions}\label{challenges-solutions}

\begin{itemize}
\tightlist
\item
  \textbf{Data Scarcity}: Use alternative data (news sentiment, Eskom
  reports) to augment limited JSE history.\\
\item
  \textbf{Overfitting}: Regularization (L1/L2 for ML) and parsimony
  checks for GARCH.\\
\item
  \textbf{Computational Cost}: Optimize code with \texttt{data.table} or
  cloud computing (AWS/GCP).
\end{itemize}

\section{Why This Matters}\label{why-this-matters}

\begin{itemize}
\tightlist
\item
  \textbf{Emerging Markets Gap}: Most volatility studies focus on
  developed markets; SA's unique risks (commodities, politics) offer
  fresh insights.\\
\item
  \textbf{Practical Value}: Helps SA investors manage risk in a
  turbulent market.
\end{itemize}

\phantomsection\label{refs}
\begin{CSLReferences}{1}{1}
\bibitem[\citeproctext]{ref-GUNNARSSON2024103221}
Gunnarsson, E.S., Isern, H.R., Kaloudis, A., Risstad, M., Vigdel, B. \&
Westgaard, S. 2024.
\href{https://doi.org/10.1016/j.irfa.2024.103221}{Prediction of realized
volatility and implied volatility indices using AI and machine learning:
A review}. \emph{International Review of Financial Analysis}. 93:103221.

\bibitem[\citeproctext]{ref-Horvath2021DeepVol}
Horvath, B., Muguruza, A. \& Tomas, M. 2021.
\href{https://doi.org/10.1080/14697688.2020.1817974}{{Deep learning
volatility: a deep neural network perspective on pricing and calibration
in (rough) volatility models}}. \emph{Quantitative Finance}.
21(1):11--27.

\bibitem[\citeproctext]{ref-Petrozziello2022DeepVol}
Petrozziello, A., Troiano, L., Serra, A., Jordanov, I., Storti, G.,
Tagliaferri, R. \& Rocca, M.L. 2022.
\href{https://doi.org/10.1007/s00500-022-07161-1}{Deep learning for
volatility forecasting in asset management}. \emph{Soft Computing}.
26(17):8553--8574.

\bibitem[\citeproctext]{ref-Wu2024InvestableAI}
Wu, E., Cheng, P., Jankovic, L. \& Kolanovic, M. 2024. \emph{Investable
AI for volatility trading deep learning model for cross asset volatility
strategies}. (Research Report). J.P. Morgan Global Quantitative \&
Derivatives Strategy. {[}Online{]}, Available:
\href{https://httpswww.jpmorganmarkets.com}{httpswww.jpmorganmarkets.com}.

\bibitem[\citeproctext]{ref-Wu2025CrossAssetVol}
Wu, E., Jankovic, L., Kaplan, B. \& Lee, T.S. 2025. \emph{Cross asset
volatility: Machine learning based trade recommendations}. (Research
Report). J.P. Morgan Global Markets Strategy. {[}Online{]}, Available:
\url{https://www.jpmorganmarkets.com}.

\bibitem[\citeproctext]{ref-ZENG2019376}
Zeng, Y. \& Klabjan, D. 2019.
\href{https://doi.org/10.1016/j.knosys.2018.08.039}{Online adaptive
machine learning based algorithm for implied volatility surface
modeling}. \emph{Knowledge-Based Systems}. 163:376--391.

\bibitem[\citeproctext]{ref-Zhang2023VolForecastML}
Zhang, C., Zhang, Y., Cucuringu, M. \& Qian, Z. 2023.
\href{https://doi.org/10.1093/jjfinec/nbad005}{Volatility forecasting
with machine learning and intraday commonality}. \emph{Journal of
Financial Econometrics}. 22(2):492--530.

\end{CSLReferences}

\bibliography{Tex/ref}





\end{document}
