\documentclass[11pt,preprint]{elsarticle}

\usepackage{lmodern}
%%%% My spacing
\usepackage{setspace}
\setstretch{1.2}
\DeclareMathSizes{12}{14}{10}{10}

% Wrap around which gives all figures included the [H] command, or places it "here". This can be tedious to code in Rmarkdown.
\usepackage{float}
\let\origfigure\figure
\let\endorigfigure\endfigure
\renewenvironment{figure}[1][2] {
    \expandafter\origfigure\expandafter[H]
} {
    \endorigfigure
}

\let\origtable\table
\let\endorigtable\endtable
\renewenvironment{table}[1][2] {
    \expandafter\origtable\expandafter[H]
} {
    \endorigtable
}


\usepackage{ifxetex,ifluatex}
\usepackage{fixltx2e} % provides \textsubscript
\ifnum 0\ifxetex 1\fi\ifluatex 1\fi=0 % if pdftex
  \usepackage[T1]{fontenc}
  \usepackage[utf8]{inputenc}
\else % if luatex or xelatex
  \ifxetex
    \usepackage{mathspec}
    \usepackage{xltxtra,xunicode}
  \else
    \usepackage{fontspec}
  \fi
  \defaultfontfeatures{Mapping=tex-text,Scale=MatchLowercase}
  \newcommand{\euro}{€}
\fi

\usepackage{amssymb, amsmath, amsthm, amsfonts}

\def\bibsection{\section*{References}} %%% Make "References" appear before bibliography


\usepackage[numbers]{natbib}

\usepackage{longtable}
\usepackage[margin=2.3cm,bottom=2cm,top=2.5cm, includefoot]{geometry}
\usepackage{fancyhdr}
\usepackage[bottom, hang, flushmargin]{footmisc}
\usepackage{graphicx}
\numberwithin{equation}{section}
\numberwithin{figure}{section}
\numberwithin{table}{section}
\setlength{\parindent}{0cm}
\setlength{\parskip}{1.3ex plus 0.5ex minus 0.3ex}
\usepackage{textcomp}
\renewcommand{\headrulewidth}{0.2pt}
\renewcommand{\footrulewidth}{0.3pt}

\usepackage{array}
\newcolumntype{x}[1]{>{\centering\arraybackslash\hspace{0pt}}p{#1}}

%%%%  Remove the "preprint submitted to" part. Don't worry about this either, it just looks better without it:
\makeatletter
\def\ps@pprintTitle{%
  \let\@oddhead\@empty
  \let\@evenhead\@empty
  \let\@oddfoot\@empty
  \let\@evenfoot\@oddfoot
}
\makeatother

 \def\tightlist{} % This allows for subbullets!

\usepackage{hyperref}
\hypersetup{breaklinks=true,
            bookmarks=true,
            colorlinks=true,
            citecolor=blue,
            urlcolor=blue,
            linkcolor=blue,
            pdfborder={0 0 0}}


% The following packages allow huxtable to work:
\usepackage{siunitx}
\usepackage{multirow}
\usepackage{hhline}
\usepackage{calc}
\usepackage{tabularx}
\usepackage{booktabs}
\usepackage{caption}


\newenvironment{columns}[1][]{}{}

\newenvironment{column}[1]{\begin{minipage}{#1}\ignorespaces}{%
\end{minipage}
\ifhmode\unskip\fi
\aftergroup\useignorespacesandallpars}

\def\useignorespacesandallpars#1\ignorespaces\fi{%
#1\fi\ignorespacesandallpars}

\makeatletter
\def\ignorespacesandallpars{%
  \@ifnextchar\par
    {\expandafter\ignorespacesandallpars\@gobble}%
    {}%
}
\makeatother


% definitions for citeproc citations
\NewDocumentCommand\citeproctext{}{}
\NewDocumentCommand\citeproc{mm}{%
\href{\#cite.\detokenize{#1}}{#2}\nocite{#1}}

\makeatletter
% allow citations to break across lines
\let\@cite@ofmt\@firstofone
% avoid brackets around text for \cite:
\def\@biblabel#1{}
\def\@cite#1#2{{#1\if@tempswa , #2\fi}}
\makeatother
\newlength{\cslhangindent}
\setlength{\cslhangindent}{1.5em}
\newlength{\csllabelwidth}
\setlength{\csllabelwidth}{3em}
\newenvironment{CSLReferences}[2] % #1 hanging-indent, #2 entry-spacing
{\begin{list}{}{%
	\setlength{\itemindent}{0pt}
	\setlength{\leftmargin}{0pt}
	\setlength{\parsep}{0pt}
	% turn on hanging indent if param 1 is 1
	\ifodd #1
	\setlength{\leftmargin}{\cslhangindent}
	\setlength{\itemindent}{-1\cslhangindent}
	\fi
	% set entry spacing
	\setlength{\itemsep}{#2\baselineskip}}}
{\end{list}}

\usepackage{calc}
\newcommand{\CSLBlock}[1]{\hfill\break\parbox[t]{\linewidth}{\strut\ignorespaces#1\strut}}
\newcommand{\CSLLeftMargin}[1]{\parbox[t]{\csllabelwidth}{\strut#1\strut}}
\newcommand{\CSLRightInline}[1]{\parbox[t]{\linewidth - \csllabelwidth}{\strut#1\strut}}
\newcommand{\CSLIndent}[1]{\hspace{\cslhangindent}#1}


\urlstyle{same}  % don't use monospace font for urls
\setlength{\parindent}{0pt}
\setlength{\parskip}{6pt plus 2pt minus 1pt}
\setlength{\emergencystretch}{3em}  % prevent overfull lines
\setcounter{secnumdepth}{5}

%%% Use protect on footnotes to avoid problems with footnotes in titles
\let\rmarkdownfootnote\footnote%
\def\footnote{\protect\rmarkdownfootnote}
\IfFileExists{upquote.sty}{\usepackage{upquote}}{}

%%% Include extra packages specified by user

%%% Hard setting column skips for reports - this ensures greater consistency and control over the length settings in the document.
%% page layout
%% paragraphs
\setlength{\baselineskip}{12pt plus 0pt minus 0pt}
\setlength{\parskip}{12pt plus 0pt minus 0pt}
\setlength{\parindent}{0pt plus 0pt minus 0pt}
%% floats
\setlength{\floatsep}{12pt plus 0 pt minus 0pt}
\setlength{\textfloatsep}{20pt plus 0pt minus 0pt}
\setlength{\intextsep}{14pt plus 0pt minus 0pt}
\setlength{\dbltextfloatsep}{20pt plus 0pt minus 0pt}
\setlength{\dblfloatsep}{14pt plus 0pt minus 0pt}
%% maths
\setlength{\abovedisplayskip}{12pt plus 0pt minus 0pt}
\setlength{\belowdisplayskip}{12pt plus 0pt minus 0pt}
%% lists
\setlength{\topsep}{10pt plus 0pt minus 0pt}
\setlength{\partopsep}{3pt plus 0pt minus 0pt}
\setlength{\itemsep}{5pt plus 0pt minus 0pt}
\setlength{\labelsep}{8mm plus 0mm minus 0mm}
\setlength{\parsep}{\the\parskip}
\setlength{\listparindent}{\the\parindent}
%% verbatim
\setlength{\fboxsep}{5pt plus 0pt minus 0pt}



\begin{document}



\begin{frontmatter}  %

\title{Thesis Proposal: Machine Learning and Volatility Models}

% Set to FALSE if wanting to remove title (for submission)




\author[Add1]{Liam Andrew Beattie}
\ead{22562435@sun.ac.za}





\address[Add1]{Stellenbosch University, South Africa}



\vspace{1cm}





\vspace{0.5cm}

\end{frontmatter}

\setcounter{footnote}{0}



%________________________
% Header and Footers
%%%%%%%%%%%%%%%%%%%%%%%%%%%%%%%%%
\pagestyle{fancy}
\chead{}
\rhead{}
\lfoot{}
\rfoot{\footnotesize Page \thepage}
\lhead{}
%\rfoot{\footnotesize Page \thepage } % "e.g. Page 2"
\cfoot{}

%\setlength\headheight{30pt}
%%%%%%%%%%%%%%%%%%%%%%%%%%%%%%%%%
%________________________

\headsep 35pt % So that header does not go over title




\section{\texorpdfstring{Introduction
\label{Introduction}}{Introduction }}\label{introduction}

Here are the papers I intend to read and give a brief summary of them:

Gunnarsson, Isern, Kaloudis, Risstad, Vigdel \& Westgaard
(\citeproc{ref-GUNNARSSON2024103221}{2024: 33}) Horvath, Muguruza \&
Tomas (\citeproc{ref-Horvath2021DeepVol}{2021}) Petrozziello, Troiano,
Serra, Jordanov, Storti, Tagliaferri \& Rocca
(\citeproc{ref-Petrozziello2022DeepVol}{2022}) Wu, Cheng, Jankovic \&
Kolanovic (\citeproc{ref-Wu2024InvestableAI}{2024}) Wu, Jankovic, Kaplan
\& Lee (\citeproc{ref-Wu2025CrossAssetVol}{2025}) Zeng \& Klabjan
(\citeproc{ref-ZENG2019376}{2019}) Zhang, Zhang, Cucuringu \& Qian
(\citeproc{ref-Zhang2023VolForecastML}{2023})

Zhang \emph{et al.} (\citeproc{ref-Zhang2023VolForecastML}{2023})

\section*{Data}\label{data}
\addcontentsline{toc}{section}{Data}

To write a comprehensive, focused quantitative econometric paper on
machine learning and volatility models with less emphasis on an
extensive literature review, you should prioritize a \textbf{clearly
defined and narrow research question} and a \textbf{rigorous
quantitative analysis} directly addressing that question. Here's how you
can structure your paper, drawing on the provided sources:

\textbf{I. Introduction and Focused Research Question:}

\begin{itemize}
\tightlist
\item
  Start with a concise introduction that immediately highlights the
  specific gap you are addressing in the literature or a particular
  problem you are investigating.
\item
  \textbf{Clearly state your focused research question(s)}. For
  instance, instead of broadly investigating if ML can forecast
  volatility better than traditional models, you might focus on:

  \begin{itemize}
  \tightlist
  \item
    The performance of a specific novel ML architecture (e.g., an
    attention-based deep learning model) in forecasting the implied
    volatility of a particular asset class (e.g., S\&P 500 options)
    compared to a specific benchmark (e.g., a specific GARCH extension
    or HAR model).
  \item
    The impact of a specific type of high-frequency intraday data (e.g.,
    order book data, transaction volume) as features in an ML model for
    short-term realized volatility forecasting for a specific set of
    stocks.
  \item
    The effectiveness of a particular XAI technique (e.g., SHAP values)
    in interpreting the predictions of a specific ML model applied to
    implied volatility surface modeling.
  \item
    The economic value (e.g., through backtesting trading strategies) of
    volatility forecasts from a specific ML model compared to a standard
    econometric model for a defined set of assets.
  \end{itemize}
\item
  Briefly mention the contribution of your focused study to the existing
  literature without an exhaustive review at this stage. You can refer
  to recent reviews to justify the relevance of your specific angle.
\item
  Outline the structure of your paper.
\end{itemize}

\textbf{II. Data:}

\begin{itemize}
\tightlist
\item
  Provide a detailed description of the specific dataset(s) you will
  use.
\item
  Clearly define your volatility measure (e.g., realized volatility
  calculated from high-frequency returns, implied volatility from option
  prices, or volatility indices like VIX).
\item
  Specify the asset class(es), time period, and sampling frequency. If
  using high-frequency data, explain the cleaning and aggregation
  methods.
\item
  Describe the predictor variables you will use, justifying their
  selection based on previous findings (cite specific papers from the
  existing literature concisely) or theoretical arguments. If using
  exogenous data, clearly state its source and frequency.
\end{itemize}

\textbf{III. Methodology:}

\begin{itemize}
\tightlist
\item
  \textbf{Clearly and concisely present the econometric and machine
  learning models} you will employ. Focus on the mathematical
  specification of the models.

  \begin{itemize}
  \tightlist
  \item
    For ML models, describe the architecture, activation functions,
    optimization algorithm (e.g., Adam), and hyperparameter tuning
    process (e.g., cross-validation).
  \item
    For benchmark econometric models (e.g., GARCH, HAR), provide their
    standard formulations.
  \end{itemize}
\item
  Explain your chosen training scheme (e.g., single asset, pooling data)
  and out-of-sample testing procedure (e.g., rolling window).
\item
  If applicable, describe any feature engineering techniques or data
  transformations you will use.
\item
  If your focus includes explainability, detail the XAI methods you will
  apply.
\item
  If you are investigating computational aspects, describe the hardware
  or software used (e.g., FPGA technology).
\end{itemize}

\textbf{IV. Evaluation Metrics:}

\begin{itemize}
\tightlist
\item
  Specify the \textbf{quantitative evaluation metrics} you will use to
  compare the forecasting performance of your models. These could
  include statistical measures like Mean Squared Error (MSE), Root Mean
  Squared Error (RMSE), Mean Absolute Percentage Error (MAPE),
  Quasi-Likelihood (QLIKE), and statistical tests for predictive
  accuracy like the Diebold-Mariano (DM) test.
\item
  If your research question involves economic significance, clearly
  outline how you will evaluate the models from an economic point of
  view (e.g., through backtesting trading strategies, Value-at-Risk
  (VaR) analysis).
\end{itemize}

\textbf{V. Empirical Results and Discussion:}

\begin{itemize}
\tightlist
\item
  Present your quantitative results in a clear and organized manner
  using tables and figures.
\item
  \textbf{Focus your discussion directly on answering your stated
  research question(s)} based on the empirical evidence.
\item
  Compare the performance of your chosen ML model(s) against the
  benchmark econometric model(s) using the selected evaluation metrics.
\item
  Analyze the statistical significance of any performance differences
  (e.g., using DM tests).
\item
  Discuss the implications of your findings in the context of the
  specific niche you are investigating. Briefly connect your results to
  the broader literature, highlighting how your focused study
  contributes or contrasts with existing findings. You can be more
  selective in the literature you discuss here, focusing on the most
  directly relevant studies.
\item
  If applicable, interpret the results of your XAI analysis, explaining
  which features are most important for your ML model's predictions.
\item
  If you conducted an economic evaluation, discuss the practical
  implications of your findings.
\end{itemize}

\textbf{VI. Conclusion and Future Work:}

\begin{itemize}
\tightlist
\item
  Summarize your main quantitative findings and directly address your
  research question(s).
\item
  Briefly reiterate the contribution of your focused study.
\item
  Suggest specific avenues for future research that build directly on
  your findings and the limitations of your study.
\end{itemize}

\textbf{Strategies for Reducing Literature Review:}

\begin{itemize}
\tightlist
\item
  \textbf{Target a Very Specific Gap:} Instead of reviewing the entire
  landscape of ML in volatility, focus on a very narrow sub-topic where
  the existing literature is limited or where a specific technique has
  not been thoroughly explored (e.g., the application of a particular
  deep learning architecture to a niche market). The literature review
  can then be more concentrated on the immediate context of your
  research question.
\item
  \textbf{Build Upon a Recent Survey:} Since comprehensive literature
  reviews exist, you can briefly summarize the state of the art based on
  these reviews and then immediately narrow down to the specific issue
  your paper addresses.
\item
  \textbf{Focus on Methodological Innovation or Application:} If your
  paper introduces a novel ML technique or applies an existing one to a
  new and underexplored area of volatility modeling, the literature
  review can be more focused on the methodological background and the
  specific context of your application, rather than a broad overview of
  all volatility models or all ML techniques.
\item
  \textbf{Assume Familiarity with Standard Models:} For well-established
  econometric models (like GARCH) and common ML algorithms (like basic
  neural networks), you can often provide a brief description and cite
  seminal works without an extensive historical review.
\end{itemize}

By adopting a focused research question and prioritizing a rigorous
quantitative analysis, you can write a comprehensive and valuable
econometric paper on machine learning and volatility models with a less
extensive literature review. The key is to be precise in your question,
thorough in your methodology and evaluation, and direct in your
discussion of the results in relation to your specific focus.

\phantomsection\label{refs}
\begin{CSLReferences}{1}{1}
\bibitem[\citeproctext]{ref-GUNNARSSON2024103221}
Gunnarsson, E.S., Isern, H.R., Kaloudis, A., Risstad, M., Vigdel, B. \&
Westgaard, S. 2024.
\href{https://doi.org/10.1016/j.irfa.2024.103221}{Prediction of realized
volatility and implied volatility indices using AI and machine learning:
A review}. \emph{International Review of Financial Analysis}. 93:103221.

\bibitem[\citeproctext]{ref-Horvath2021DeepVol}
Horvath, B., Muguruza, A. \& Tomas, M. 2021.
\href{https://doi.org/10.1080/14697688.2020.1817974}{{Deep learning
volatility: a deep neural network perspective on pricing and calibration
in (rough) volatility models}}. \emph{Quantitative Finance}.
21(1):11--27.

\bibitem[\citeproctext]{ref-Petrozziello2022DeepVol}
Petrozziello, A., Troiano, L., Serra, A., Jordanov, I., Storti, G.,
Tagliaferri, R. \& Rocca, M.L. 2022.
\href{https://doi.org/10.1007/s00500-022-07161-1}{Deep learning for
volatility forecasting in asset management}. \emph{Soft Computing}.
26(17):8553--8574.

\bibitem[\citeproctext]{ref-Wu2024InvestableAI}
Wu, E., Cheng, P., Jankovic, L. \& Kolanovic, M. 2024. \emph{Investable
AI for volatility trading deep learning model for cross asset volatility
strategies}. (Research Report). J.P. Morgan Global Quantitative \&
Derivatives Strategy. {[}Online{]}, Available:
\href{https://httpswww.jpmorganmarkets.com}{httpswww.jpmorganmarkets.com}.

\bibitem[\citeproctext]{ref-Wu2025CrossAssetVol}
Wu, E., Jankovic, L., Kaplan, B. \& Lee, T.S. 2025. \emph{Cross asset
volatility: Machine learning based trade recommendations}. (Research
Report). J.P. Morgan Global Markets Strategy. {[}Online{]}, Available:
\url{https://www.jpmorganmarkets.com}.

\bibitem[\citeproctext]{ref-ZENG2019376}
Zeng, Y. \& Klabjan, D. 2019.
\href{https://doi.org/10.1016/j.knosys.2018.08.039}{Online adaptive
machine learning based algorithm for implied volatility surface
modeling}. \emph{Knowledge-Based Systems}. 163:376--391.

\bibitem[\citeproctext]{ref-Zhang2023VolForecastML}
Zhang, C., Zhang, Y., Cucuringu, M. \& Qian, Z. 2023.
\href{https://doi.org/10.1093/jjfinec/nbad005}{Volatility forecasting
with machine learning and intraday commonality}. \emph{Journal of
Financial Econometrics}. 22(2):492--530.

\end{CSLReferences}

\bibliography{Tex/ref}





\end{document}
