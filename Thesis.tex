\documentclass[11pt,preprint]{elsarticle}

\usepackage{lmodern}
%%%% My spacing
\usepackage{setspace}
\setstretch{1.2}
\DeclareMathSizes{12}{14}{10}{10}

% Wrap around which gives all figures included the [H] command, or places it "here". This can be tedious to code in Rmarkdown.
\usepackage{float}
\let\origfigure\figure
\let\endorigfigure\endfigure
\renewenvironment{figure}[1][2] {
    \expandafter\origfigure\expandafter[H]
} {
    \endorigfigure
}

\let\origtable\table
\let\endorigtable\endtable
\renewenvironment{table}[1][2] {
    \expandafter\origtable\expandafter[H]
} {
    \endorigtable
}


\usepackage{ifxetex,ifluatex}
\usepackage{fixltx2e} % provides \textsubscript
\ifnum 0\ifxetex 1\fi\ifluatex 1\fi=0 % if pdftex
  \usepackage[T1]{fontenc}
  \usepackage[utf8]{inputenc}
\else % if luatex or xelatex
  \ifxetex
    \usepackage{mathspec}
    \usepackage{xltxtra,xunicode}
  \else
    \usepackage{fontspec}
  \fi
  \defaultfontfeatures{Mapping=tex-text,Scale=MatchLowercase}
  \newcommand{\euro}{€}
\fi

\usepackage{amssymb, amsmath, amsthm, amsfonts}

\def\bibsection{\section*{References}} %%% Make "References" appear before bibliography


\usepackage[numbers]{natbib}

\usepackage{longtable}
\usepackage[margin=2.3cm,bottom=2cm,top=2.5cm, includefoot]{geometry}
\usepackage{fancyhdr}
\usepackage[bottom, hang, flushmargin]{footmisc}
\usepackage{graphicx}
\numberwithin{equation}{section}
\numberwithin{figure}{section}
\numberwithin{table}{section}
\setlength{\parindent}{0cm}
\setlength{\parskip}{1.3ex plus 0.5ex minus 0.3ex}
\usepackage{textcomp}
\renewcommand{\headrulewidth}{0.2pt}
\renewcommand{\footrulewidth}{0.3pt}

\usepackage{array}
\newcolumntype{x}[1]{>{\centering\arraybackslash\hspace{0pt}}p{#1}}

%%%%  Remove the "preprint submitted to" part. Don't worry about this either, it just looks better without it:
\makeatletter
\def\ps@pprintTitle{%
  \let\@oddhead\@empty
  \let\@evenhead\@empty
  \let\@oddfoot\@empty
  \let\@evenfoot\@oddfoot
}
\makeatother

 \def\tightlist{} % This allows for subbullets!

\usepackage{hyperref}
\hypersetup{breaklinks=true,
            bookmarks=true,
            colorlinks=true,
            citecolor=blue,
            urlcolor=blue,
            linkcolor=blue,
            pdfborder={0 0 0}}


% The following packages allow huxtable to work:
\usepackage{siunitx}
\usepackage{multirow}
\usepackage{hhline}
\usepackage{calc}
\usepackage{tabularx}
\usepackage{booktabs}
\usepackage{caption}


\newenvironment{columns}[1][]{}{}

\newenvironment{column}[1]{\begin{minipage}{#1}\ignorespaces}{%
\end{minipage}
\ifhmode\unskip\fi
\aftergroup\useignorespacesandallpars}

\def\useignorespacesandallpars#1\ignorespaces\fi{%
#1\fi\ignorespacesandallpars}

\makeatletter
\def\ignorespacesandallpars{%
  \@ifnextchar\par
    {\expandafter\ignorespacesandallpars\@gobble}%
    {}%
}
\makeatother


% definitions for citeproc citations
\NewDocumentCommand\citeproctext{}{}
\NewDocumentCommand\citeproc{mm}{%
\href{\#cite.\detokenize{#1}}{#2}\nocite{#1}}

\makeatletter
% allow citations to break across lines
\let\@cite@ofmt\@firstofone
% avoid brackets around text for \cite:
\def\@biblabel#1{}
\def\@cite#1#2{{#1\if@tempswa , #2\fi}}
\makeatother
\newlength{\cslhangindent}
\setlength{\cslhangindent}{1.5em}
\newlength{\csllabelwidth}
\setlength{\csllabelwidth}{3em}
\newenvironment{CSLReferences}[2] % #1 hanging-indent, #2 entry-spacing
{\begin{list}{}{%
	\setlength{\itemindent}{0pt}
	\setlength{\leftmargin}{0pt}
	\setlength{\parsep}{0pt}
	% turn on hanging indent if param 1 is 1
	\ifodd #1
	\setlength{\leftmargin}{\cslhangindent}
	\setlength{\itemindent}{-1\cslhangindent}
	\fi
	% set entry spacing
	\setlength{\itemsep}{#2\baselineskip}}}
{\end{list}}

\usepackage{calc}
\newcommand{\CSLBlock}[1]{\hfill\break\parbox[t]{\linewidth}{\strut\ignorespaces#1\strut}}
\newcommand{\CSLLeftMargin}[1]{\parbox[t]{\csllabelwidth}{\strut#1\strut}}
\newcommand{\CSLRightInline}[1]{\parbox[t]{\linewidth - \csllabelwidth}{\strut#1\strut}}
\newcommand{\CSLIndent}[1]{\hspace{\cslhangindent}#1}


\urlstyle{same}  % don't use monospace font for urls
\setlength{\parindent}{0pt}
\setlength{\parskip}{6pt plus 2pt minus 1pt}
\setlength{\emergencystretch}{3em}  % prevent overfull lines
\setcounter{secnumdepth}{5}

%%% Use protect on footnotes to avoid problems with footnotes in titles
\let\rmarkdownfootnote\footnote%
\def\footnote{\protect\rmarkdownfootnote}
\IfFileExists{upquote.sty}{\usepackage{upquote}}{}

%%% Include extra packages specified by user

%%% Hard setting column skips for reports - this ensures greater consistency and control over the length settings in the document.
%% page layout
%% paragraphs
\setlength{\baselineskip}{12pt plus 0pt minus 0pt}
\setlength{\parskip}{12pt plus 0pt minus 0pt}
\setlength{\parindent}{0pt plus 0pt minus 0pt}
%% floats
\setlength{\floatsep}{12pt plus 0 pt minus 0pt}
\setlength{\textfloatsep}{20pt plus 0pt minus 0pt}
\setlength{\intextsep}{14pt plus 0pt minus 0pt}
\setlength{\dbltextfloatsep}{20pt plus 0pt minus 0pt}
\setlength{\dblfloatsep}{14pt plus 0pt minus 0pt}
%% maths
\setlength{\abovedisplayskip}{12pt plus 0pt minus 0pt}
\setlength{\belowdisplayskip}{12pt plus 0pt minus 0pt}
%% lists
\setlength{\topsep}{10pt plus 0pt minus 0pt}
\setlength{\partopsep}{3pt plus 0pt minus 0pt}
\setlength{\itemsep}{5pt plus 0pt minus 0pt}
\setlength{\labelsep}{8mm plus 0mm minus 0mm}
\setlength{\parsep}{\the\parskip}
\setlength{\listparindent}{\the\parindent}
%% verbatim
\setlength{\fboxsep}{5pt plus 0pt minus 0pt}



\begin{document}



\begin{frontmatter}  %

\title{Thesis Proposal - Improving Equity Volatility Forecasts for
Emerging Markets: A Comparative Study of AI and Econometric Approaches
Using South African Data}

% Set to FALSE if wanting to remove title (for submission)




\author[Add1]{Liam Andrew Beattie}
\ead{22562435@sun.ac.za}





\address[Add1]{Stellenbosch University, South Africa}


\begin{abstract}
\small{
Testing whether modern AI methods can predict stock market swings in
South Africa---driven by issues like power cuts and mining
reliance---better than older models.
}
\end{abstract}

\vspace{1cm}





\vspace{0.5cm}

\end{frontmatter}

\setcounter{footnote}{0}



%________________________
% Header and Footers
%%%%%%%%%%%%%%%%%%%%%%%%%%%%%%%%%
\pagestyle{fancy}
\chead{}
\rhead{}
\lfoot{}
\rfoot{\footnotesize Page \thepage}
\lhead{}
%\rfoot{\footnotesize Page \thepage } % "e.g. Page 2"
\cfoot{}

%\setlength\headheight{30pt}
%%%%%%%%%%%%%%%%%%%%%%%%%%%%%%%%%
%________________________

\headsep 35pt % So that header does not go over title




\section{\texorpdfstring{Introduction
\label{Introduction}}{Introduction }}\label{introduction}

Predicting stock market volatility---how wildly prices swing---is
crucial for investors and policymakers, especially in emerging markets
like South Africa. The Johannesburg Stock Exchange (JSE) faces unique
challenges: its economy relies heavily on gold and platinum exports,
suffers frequent power blackouts (like Eskom's load-shedding), and
grapples with political uncertainty. Traditional forecasting models,
such as GARCH or HAR, are designed for stable markets and often miss
these local, real-world risks. This raises a critical question:
\emph{Can machine learning (ML) models like LSTM and XGBoost predict JSE
volatility more accurately by incorporating these South African-specific
factors?}

To answer this, the study will analyze JSE data (e.g., Top 40 index) and
test whether ML models outperform traditional methods when local drivers
are included. The process starts with preparing high-frequency trading
data, calculating volatility, and integrating SA-specific variables like
commodity prices and load-shedding schedules. Models will be trained
using time-series techniques to avoid overfitting, including hybrid
approaches that combine traditional methods (e.g., HAR) with ML (e.g.,
LSTM) to balance structure and flexibility.

Accuracy will be tested using standard metrics like mean squared error
(MSE) and real-world scenarios, such as predicting losses during power
blackouts or political crises. A key focus is transparency: tools like
SHAP values will explain why models make predictions---for example,
revealing whether load-shedding matters more than gold prices during
crises.

Results could help South African investors hedge risks more effectively
and guide policymakers in stabilizing markets during infrastructure or
political shocks. By focusing on SA's unique challenges, this study aims
to show whether modern ML tools can outperform traditional models in
messy, real-world emerging markets---where global theories often fall
short.

\begin{quote}
What follows is the thesis outline, followed by Research Design \& Data
Collection, then Model Selection \& Training, then Evaluation Framework,
and finally Hopeful Contributions
\end{quote}

\section{Thesis Structure:}\label{thesis-structure}

\begin{itemize}
\tightlist
\item
  Introduction

  \begin{itemize}
  \tightlist
  \item
    SA's volatility challenges; ML vs.~traditional debate
    (\citeproc{ref-GUNNARSSON2024103221}{Gunnarsson, Isern, Kaloudis,
    Risstad, Vigdel \& Westgaard, 2024};
    \citeproc{ref-Wu2024InvestableAI}{Wu, Cheng, Jankovic \& Kolanovic,
    2024})
  \end{itemize}
\item
  Literature Review

  \begin{itemize}
  \tightlist
  \item
    ML in volatility forecasting
    (\citeproc{ref-GUNNARSSON2024103221}{Gunnarsson \emph{et al.},
    2024}; \citeproc{ref-Horvath2021DeepVol}{Horvath, Muguruza \& Tomas,
    2021}), emerging market gaps
    (\citeproc{ref-Zhang2023VolForecastML}{Zhang, Zhang, Cucuringu \&
    Qian, 2023}).
  \end{itemize}
\item
  Methodology

  \begin{itemize}
  \tightlist
  \item
    Data preprocessing
    (\citeproc{ref-Petrozziello2022DeepVol}{Petrozziello, Troiano,
    Serra, Jordanov, Storti, Tagliaferri \& Rocca, 2022};
    \citeproc{ref-Zhang2023VolForecastML}{Zhang \emph{et al.}, 2023})
  \item
    Model design (\citeproc{ref-GUNNARSSON2024103221}{Gunnarsson
    \emph{et al.}, 2024}; \citeproc{ref-Horvath2021DeepVol}{Horvath
    \emph{et al.}, 2021}; \citeproc{ref-Wu2024InvestableAI}{Wu \emph{et
    al.}, 2024})\\
  \item
    Adaptive learning (\citeproc{ref-ZENG2019376}{Zeng \& Klabjan,
    2019}).
  \end{itemize}
\item
  Results

  \begin{itemize}
  \tightlist
  \item
    Statistical/economic metrics\\
  \item
    SHAP interpretation (\citeproc{ref-GUNNARSSON2024103221}{Gunnarsson
    \emph{et al.}, 2024}; \citeproc{ref-Wu2025CrossAssetVol}{Wu,
    Jankovic, Kaplan \& Lee, 2025})
  \end{itemize}
\item
  Discussion

  \begin{itemize}
  \tightlist
  \item
    ML flexibility in SA's context vs.~developed markets
    (\citeproc{ref-GUNNARSSON2024103221}{Gunnarsson \emph{et al.},
    2024}; \citeproc{ref-Zhang2023VolForecastML}{Zhang \emph{et al.},
    2023}).\\
  \item
    Policy/investor implications (\citeproc{ref-Wu2025CrossAssetVol}{Wu
    \emph{et al.}, 2025}; \citeproc{ref-ZENG2019376}{Zeng \& Klabjan,
    2019}).
  \end{itemize}
\item
  Conclusion

  \begin{itemize}
  \tightlist
  \item
    Synthesis of findings; future work (e.g., rough volatility with JSE
    options data).
  \end{itemize}
\end{itemize}

\section{Research Design \& Data
Collection}\label{research-design-data-collection}

\textbf{Data sources}

\begin{itemize}
\tightlist
\item
  Core Data: JSE Top 40 high-frequency returns (15-minute intervals,
  adjusted for liquidity constraints per Gunnarsson \emph{et al.}
  (\citeproc{ref-GUNNARSSON2024103221}{2024}) and Petrozziello \emph{et
  al.} (\citeproc{ref-Petrozziello2022DeepVol}{2022}).\\
\item
  Exogenous Variables:

  \begin{itemize}
  \tightlist
  \item
    Commodity prices (platinum, gold) -- Gunnarsson \emph{et al.}
    (\citeproc{ref-GUNNARSSON2024103221}{2024}), Wu \emph{et al.}
    (\citeproc{ref-Wu2024InvestableAI}{2024}).\\
  \item
    Eskom load-shedding schedules (categorical/time-series) --
    (\citeproc{ref-Petrozziello2022DeepVol}{Petrozziello \emph{et al.},
    2022}; \citeproc{ref-Zhang2023VolForecastML}{Zhang \emph{et al.},
    2023}).\\
  \item
    SA political risk index (text-based scores) -- Gunnarsson \emph{et
    al.} (\citeproc{ref-GUNNARSSON2024103221}{2024}).
    \emph{(Realistically this may be hard to do but I can attempt)}\\
  \end{itemize}
\item
  Benchmarks: Realized Volatility (RV) using subsampled kernels
  (\citeproc{ref-Petrozziello2022DeepVol}{Petrozziello \emph{et al.},
  2022}; \citeproc{ref-Zhang2023VolForecastML}{Zhang \emph{et al.},
  2023}) to mitigate microstructure noise.
\end{itemize}

\textbf{Preprocessing:}

\begin{itemize}
\tightlist
\item
  RV Calculation: Use 15-minute intervals (not 5-minute, aligning with
  Gunnarsson \emph{et al.} (\citeproc{ref-GUNNARSSON2024103221}{2024})
  liquidity caveat).\\
\item
  Feature Engineering:

  \begin{itemize}
  \tightlist
  \item
    Lagged RV components (HAR-RV structure) -- Gunnarsson \emph{et al.}
    (\citeproc{ref-GUNNARSSON2024103221}{2024}), Petrozziello \emph{et
    al.} (\citeproc{ref-Petrozziello2022DeepVol}{2022})
  \item
    Grid Representation: Convert RV and exogenous variables into
    time-strike matrices (\citeproc{ref-Horvath2021DeepVol}{Horvath
    \emph{et al.}, 2021}) but limit grid complexity to 3 channels (RV,
    commodities, load-shedding) to avoid overfitting.\\
  \item
    Diurnal adjustment and winsorization (Zhang \emph{et al.}
    (\citeproc{ref-Zhang2023VolForecastML}{2023})) for intraday returns.
  \end{itemize}
\end{itemize}

\section{Model Selection \& Training}\label{model-selection-training}

\textbf{Models}

\begin{enumerate}
\def\labelenumi{\arabic{enumi}.}
\tightlist
\item
  Traditional:

  \begin{itemize}
  \tightlist
  \item
    GARCH(1,1), HAR-RV ( Gunnarsson \emph{et al.}
    (\citeproc{ref-GUNNARSSON2024103221}{2024}), Petrozziello \emph{et
    al.} (\citeproc{ref-Petrozziello2022DeepVol}{2022})).\\
  \end{itemize}
\item
  Machine Learning:

  \begin{itemize}
  \tightlist
  \item
    LSTM: Stacked architecture with dropout (Petrozziello \emph{et al.}
    (\citeproc{ref-Petrozziello2022DeepVol}{2022})), attention
    mechanisms for SA factors (Wu \emph{et al.}
    (\citeproc{ref-Wu2024InvestableAI}{2024}); Wu \emph{et al.}
    (\citeproc{ref-Wu2025CrossAssetVol}{2025})).\\
  \item
    XGBoost: Feature importance analysis via SHAP ( Gunnarsson \emph{et
    al.} (\citeproc{ref-GUNNARSSON2024103221}{2024})).\\
  \end{itemize}
\item
  Hybrid: HAR-LSTM (HAR-RV inputs + LSTM layers) ( Gunnarsson \emph{et
  al.} (\citeproc{ref-GUNNARSSON2024103221}{2024}), Horvath \emph{et
  al.} (\citeproc{ref-Horvath2021DeepVol}{2021})).\\
\item
  Adaptive Online Learning: Incremental LSTM/XGBoost updates (Zeng \&
  Klabjan (\citeproc{ref-ZENG2019376}{2019})) for crisis responsiveness.
\end{enumerate}

\textbf{Training Protocol}

\begin{itemize}
\tightlist
\item
  Time-Series CV: Rolling window validation (70\% train, 30\% test)
  (Petrozziello \emph{et al.}
  (\citeproc{ref-Petrozziello2022DeepVol}{2022}), Gunnarsson \emph{et
  al.} (\citeproc{ref-GUNNARSSON2024103221}{2024})).\\
\item
  Dynamic Feature Management: Retain top 10 features via SHAP-driven FVS
  (Zeng \& Klabjan (\citeproc{ref-ZENG2019376}{2019})).\\
\item
  Hardware: GPU acceleration for LSTM (Horvath \emph{et al.}
  (\citeproc{ref-Horvath2021DeepVol}{2021})) but avoid FPGA (\emph{due
  to feasibility concerns}).
\end{itemize}

\section{Evaluation Framework}\label{evaluation-framework}

\textbf{Metrics:}

\begin{itemize}
\tightlist
\item
  Statistical: MSE, QLIKE (Gunnarsson \emph{et al.}
  (\citeproc{ref-GUNNARSSON2024103221}{2024}), Petrozziello \emph{et
  al.} (\citeproc{ref-Petrozziello2022DeepVol}{2022}), Zhang \emph{et
  al.} (\citeproc{ref-Zhang2023VolForecastML}{2023})).\\
\item
  Economic: VaR, Expected Shortfall (Gunnarsson \emph{et al.}
  (\citeproc{ref-GUNNARSSON2024103221}{2024})), Sharpe/Sortino ratios
  (Wu \emph{et al.} (\citeproc{ref-Wu2024InvestableAI}{2024}); Wu
  \emph{et al.} (\citeproc{ref-Wu2025CrossAssetVol}{2025})).\\
\item
  Robustness:

  \begin{itemize}
  \tightlist
  \item
    Diebold-Mariano tests (Petrozziello \emph{et al.}
    (\citeproc{ref-Petrozziello2022DeepVol}{2022})).\\
  \item
    Regime-specific analysis (load-shedding stages, political crises) (
    Zhang \emph{et al.} (\citeproc{ref-Zhang2023VolForecastML}{2023}),
    Wu \emph{et al.} (\citeproc{ref-Wu2024InvestableAI}{2024})).
  \end{itemize}
\end{itemize}

\textbf{Interpretability:}

\begin{itemize}
\tightlist
\item
  SHAP Values: Track evolving importance of SA factors (e.g.,
  load-shedding's impact during crises) (Gunnarsson \emph{et al.}
  (\citeproc{ref-GUNNARSSON2024103221}{2024}), Zeng \& Klabjan
  (\citeproc{ref-ZENG2019376}{2019})).\\
\item
  Crisis-Tagged Plots: Overlay volatility forecasts with
  load-shedding/political events (Gunnarsson \emph{et al.}
  (\citeproc{ref-GUNNARSSON2024103221}{2024}), Zhang \emph{et al.}
  (\citeproc{ref-Zhang2023VolForecastML}{2023})).
\end{itemize}

\section{Hopeful Contributions}\label{hopeful-contributions}

\begin{enumerate}
\def\labelenumi{\arabic{enumi}.}
\tightlist
\item
  Emerging Market Insights:

  \begin{itemize}
  \tightlist
  \item
    Demonstrate ML's superiority in capturing SA-specific drivers
    (commodities, load-shedding) vs.~GARCH/HAR.\\
  \end{itemize}
\item
  Practical Tools:

  \begin{itemize}
  \tightlist
  \item
    Crisis-response dashboards with SHAP-driven volatility forecasts for
    policymakers.\\
  \end{itemize}
\item
  Methodological Advancements:

  \begin{itemize}
  \tightlist
  \item
    Hybrid HAR-LSTM model optimized for thin liquidity.
  \end{itemize}
\end{enumerate}

\newpage

\section*{References}\label{references}
\addcontentsline{toc}{section}{References}

\phantomsection\label{refs}
\begin{CSLReferences}{1}{1}
\bibitem[\citeproctext]{ref-GUNNARSSON2024103221}
Gunnarsson, E.S., Isern, H.R., Kaloudis, A., Risstad, M., Vigdel, B. \&
Westgaard, S. 2024.
\href{https://doi.org/10.1016/j.irfa.2024.103221}{Prediction of realized
volatility and implied volatility indices using AI and machine learning:
A review}. \emph{International Review of Financial Analysis}. 93:103221.

\bibitem[\citeproctext]{ref-Horvath2021DeepVol}
Horvath, B., Muguruza, A. \& Tomas, M. 2021.
\href{https://doi.org/10.1080/14697688.2020.1817974}{{Deep learning
volatility: a deep neural network perspective on pricing and calibration
in (rough) volatility models}}. \emph{Quantitative Finance}.
21(1):11--27.

\bibitem[\citeproctext]{ref-Petrozziello2022DeepVol}
Petrozziello, A., Troiano, L., Serra, A., Jordanov, I., Storti, G.,
Tagliaferri, R. \& Rocca, M.L. 2022.
\href{https://doi.org/10.1007/s00500-022-07161-1}{Deep learning for
volatility forecasting in asset management}. \emph{Soft Computing}.
26(17):8553--8574.

\bibitem[\citeproctext]{ref-Wu2024InvestableAI}
Wu, E., Cheng, P., Jankovic, L. \& Kolanovic, M. 2024. \emph{Investable
AI for volatility trading deep learning model for cross asset volatility
strategies}. (Research Report). J.P. Morgan Global Quantitative \&
Derivatives Strategy. {[}Online{]}, Available:
\href{https://httpswww.jpmorganmarkets.com}{httpswww.jpmorganmarkets.com}.

\bibitem[\citeproctext]{ref-Wu2025CrossAssetVol}
Wu, E., Jankovic, L., Kaplan, B. \& Lee, T.S. 2025. \emph{Cross asset
volatility: Machine learning based trade recommendations}. (Research
Report). J.P. Morgan Global Markets Strategy. {[}Online{]}, Available:
\url{https://www.jpmorganmarkets.com}.

\bibitem[\citeproctext]{ref-ZENG2019376}
Zeng, Y. \& Klabjan, D. 2019.
\href{https://doi.org/10.1016/j.knosys.2018.08.039}{Online adaptive
machine learning based algorithm for implied volatility surface
modeling}. \emph{Knowledge-Based Systems}. 163:376--391.

\bibitem[\citeproctext]{ref-Zhang2023VolForecastML}
Zhang, C., Zhang, Y., Cucuringu, M. \& Qian, Z. 2023.
\href{https://doi.org/10.1093/jjfinec/nbad005}{Volatility forecasting
with machine learning and intraday commonality}. \emph{Journal of
Financial Econometrics}. 22(2):492--530.

\end{CSLReferences}

\bibliography{Tex/ref}





\end{document}
